% type
\documentclass{article}

% greek characters (among others)
\usepackage[T1]{fontenc}

% format
\usepackage[letterpaper, margin=1.5cm]{geometry}

% font
\renewcommand{\familydefault}{\sfdefault}

% code
\usepackage{minted}
\usemintedstyle{emacs}

% some math utils
\usepackage{amsmath}

% header
\title{
    Práctica 4 \\
    Inferencia de tipos
}

\author{
    Sandra del Mar Soto Corderi \\
    Edgar Quiroz Castañeda
}

\begin{document}
    \maketitle

    \section{Desarrollo}

    \subsection{Tipos}

    Lo relacionado al módulo \texttt{Type}. La creación de los tipos y los alias de tipos fue directa.

    \begin{minted}{haskell}
        -- | Se extiende la categoría de tipos.
        type Identifier = Int
        infix :->

        data Type = T Identifier
            | Integer | Boolean
            | Type :-> Type deriving (Show, Eq)
            
        type Substitution = [(Identifier, Type)]
    \end{minted}

    \subsubsection{\texttt{vars}}

    El tipo \texttt{Type} es un tipo inductivo. Así que la función se definió
    recursivamente.
    Simplemente, si se tiene una variable, se agrega al conjunto de variables.
    Si se tiene un tipo concreto (\texttt{Boolean, Integer}), no se hace nada.
    Si se tiene una función, se hace recursión sobre las subestrucutras.

    \begin{minted}{haskell}
        -- | Devuelve el conjunto de variables de tipo
        vars :: Type -> [Identifier]
        vars (T t) = [t]
        vars (t1 :-> t2) = List.union (vars t1) (vars t2)
        vars _ = []
    \end{minted}

    \subsubsection{\texttt{subst}}

    En este caso hay que hacer una doble inducción sobre la sustitución (que es
    una lista) y el tipo.

    Primero se hace recursión sobre el tipo.
    Usando el mismo principio inductivo, si se tiene una variable, entonces se
    busca en la sustitución.
    Si se tiene un tipo concreto, no se hace nada.
    Si se tiene una función, se hace recursión.

    Luego, cuando se tiene una variable de tipo, se busca la variable
    recursivamente en la sustitución.

    \begin{minted}{haskell}
        -- | Aplica la sustitución a un tipo dado
        subst :: Type -> Substitution -> Type
        subst e@(T t) s = 
            case s of
            [] -> e
            ((x, t'): ss) -> 
                if x == t then t' else subst e ss
        subst (t1 :-> t2) s = (subst t1 s) :-> (subst t2 s)
        subst t _ = t
    \end{minted}

    \subsubsection{\texttt{comp}}

    Traducción literal de la definición de composición de sustituciones.

    Esto es, que $\sigma \circ \rho$ corresponde a
    
    \[
        (\sigma \circ \rho)(x) = 
        \begin{cases}
            \rho (\sigma (x)) \text{ si }  x \in dom(\sigma) \\
            \rho (x) \text{ en otro caso }
        \end{cases}
    \]

    Si se piensa en las sustituciones como listas (como en nuestra
    implementación) entonces esto corresponde a

    \begin{itemize}
        \item Remplazar cada tupla $(x, e) \in \sigma$ por $(x, e_{\rho})$
        \item Añadir las tuplas de $\rho$ cuyas variables no estaban definidas
        para $\sigma$.
    \end{itemize}

    En código esto sería

    \begin{minted}{haskell}
        -- | Realiza la composición de dos sustituciones
        comp :: Substitution -> Substitution -> Substitution
        comp s1 s2 = 
            List.union
            [(x, subst t s2) | (x, t) <- s1] 
            [(x, t) | (x, t) <- s2, List.notElem x [y | (y, _) <- s1]]
    \end{minted}

    \subsubsection{\texttt{simpl}}

    Para esto se definió una función auxiliar \texttt{redundant} que indica si
    una sustitución es redundante.

    Entonces, para quitar sustituciones redundantes sólo hay que aplicar un
    filtro usando esta función.

    \begin{minted}{haskell}
        -- | Elimina sustituciones redundantes
        simpl:: Substitution -> Substitution
        simpl s = filter (\x -> not (redundant x)) s
      
        -- | Verifica si una tupla de una sustitución es reduntante. Auxiliar.
        redundant :: (Identifier, Type) -> Bool
        redundant (i, T t) = i == t
        redundant _ = False
    \end{minted}

    \subsection{Inferencia}
    
    Lo relacionado al módulo \texttt{Static}. Los tipos y alias utilizados se
    crearon tal cuál se especificó en la descripción de la práctica.

    \begin{minted}{haskell}
        -- | Definiendo el contexto para tipos de variables
        type Ctxt = [(Sintax.Identifier, Type.Type)] 

        -- | Definiendo las restricciones para inferir los tipos
        type Constraint = [(Type.Type, Type.Type)]
    \end{minted}

    \subsubsection{\texttt{subst}}

    Únicamente se realizó la sustitución sobre cada una de las tuplas del
    contexto usando la función \texttt{Type.subst} definida anteriormente.

    \begin{minted}{haskell}
        -- | Realiza sustituciones de variables de tipo
        subst :: Ctxt -> Type.Substitution -> Ctxt
        subst [] _ = []
        subst ((x, t): cs) s = (x, Type.subst t s) : subst cs s
    \end{minted}

    \subsubsection{\texttt{find}}

    Se realiza la búsqueda recusrivamente sobre el contexto. De no encontrar
    nada, se regresa \texttt{Nothing}.

    \begin{minted}{haskell}
        -- | Busca el tipo de una variable en un contexto
        find :: Sintax.Identifier -> Ctxt -> Maybe Type.Type
        find _ [] = Nothing
        find x ((y, t) : cs) = 
            if x == y 
            then Just t
            else find x cs
    \end{minted}

    \subsubsection{\texttt{fresh}}

    Para esto, se usó la función \texttt{Type.vars} que regresa los
    identificadores de las variables en una expresión.

    De entre ellos, se obtuvo el más grande, y se sumó uno para obtener un
    identificador que no estaba presente en ese conjunto.

    \begin{minted}{haskell}
        -- | Obtiene una variable que no figure en la lista
        fresh :: [Type.Type] -> Type.Type
        fresh s = 
            Type.T ((foldl max 0 (foldr (\ x y -> List.union (Type.vars x) y) [] s)) + 1)
    \end{minted}

    \subsubsection{\texttt{infer'}}

    Como está especificado en la descripción de la práctica, esta función está
    dividida en seis partes principales, cada una expresada con un juicio.

    Cada parte está compuesta por varias instancias de la regla correspondiente.
    Al ser todos los casos análogos, sólo se va a describir el caso general de
    cada sección.

    \begin{itemize}
        \item Variables

        Se obtiene un nuevo tipo para la variable, la cuál se agrega al catálogo
        de variables, así como al contexto. 
        
        Y se añade un conjunto de restricciones vacío.

        \begin{minted}{haskell}
            -- varialbles
            infer' (nv, (Sintax.V x)) = 
                let t = fresh nv; nv' = nv `List.union` [t]
                    in (nv' ,[(x, t)] ,t , [])
        \end{minted}

        \item Constantes

        Sólo se da directamente el tipo. No se modificada nada más.

        \begin{minted}{haskell}
            -- T
            infer' (nv, (Sintax.T e)) = (nv, [], Type.T, [])
        \end{minted}

        \item Funciones

        Primero, es necesario tener el tipo, el contexto, las restricciones y el
        catálogo de tipos del cuerpo de la función. Esto se logra con una
        llamada recursiva.

        Luego, se necesita el tipo de la variable ligada.
        
        Si está en el cuerpo, entonces su tipo está en el contexto. Así que sólo
        hay que buscarlo y borrarlo. 
        
        Si no figuraba en el cuerpo, entonces no estará en el contexto. Así que
        habrá que crear un nuevo tipo, y agregarlo al catálogo de tipos. No se 
        debe modificar nada más.

        En ambos casos, el tipo de la función es un tipo función que recibe el
        tipo de la variable ligada y regresa el tipo del cuerpo de la función.

        \begin{minted}{haskell}
            -- functions
            infer' (nv, (Sintax.Fn x e)) = 
            let (nv', g, t, r) = infer' (nv, e)
                in case find x g of
                    Just t' -> (nv', filter (\(i, t) -> i /= x) g, t' Type.:-> t, r)
                    Nothing -> 
                    let t' = fresh nv'; nv'' = nv' `List.union` [t']
                        in (nv'', g, t' Type.:-> t, r)
        \end{minted}

        \item Operadores

        Primero, se necesita el tipo, restricciones, contexto y catálogo de tipo
        de los operados, así como el tipo que regresa la operación.

        Y se debe agregar a las restricciones que todos los diferentes tipos que 
        pueda tener una variable en realidad son el mismo.

        El catálogo de tipos es la unión de los catálogos. El contexto en la
        unión de los contextos. Las restricciones son la unión de las
        restricciones, pero además hay que agregar que todos los operandos son
        del tipo requerido.

        Entonces, si se tienen $e_1, \dots, e_n$ operandos y una función $F$ que
        regresa un tipo $T$ y que recibe $n$ operandos de tipos 
        $U_1, \dots, U_1$. Entonces, el código para procesar esta función es 

        \begin{minted}{haskell}
            -- op
            infer' (nv, (Sintax.T e1 ... en)) = 
                let (nv1, g1, t1, r1) = infer' (nv, e1);
                    ...;
                    (nvn, gn, tn, rn) = infer' (nv{n-1}, en);
                    s = [(s1, s2) | (x, s1) <- g1, (y, s2) <- g2, x == y]
                    `List.union`
                        ...
                    `List.union`
                        [(s{n-1}, sn) | (x, s{n-1}) <- g{n-1}, (y, sn) <- gn, x == y];
                    in (
                        nvn, 
                        g1 `List.union` ... `List.union` gn, 
                        T, 
                        r1 `List.union` ... `List.union` rn `List.union` s 
                        `List.union` [(t1, U1), ..., (tn, Un)]
                        )
        \end{minted}

        Hay que notar que se está omitiendo conseguir los tipo $T$ y $U_i$, lo
        cuál varia según el operador particular. 

        \item Aplicaciones

        Se requiere toda la información de las dos subexpresiones, lo cuál se
        obtiene con llamadas recursivas.

        Como en la mayoría de los casos, hay que añadir como restricciones que
        todos los posibles tipos de una variable son en realidad el mismo.

        Se requiere un nuevo tipo que represente el tipo que regresa la
        aplicación.

        Y finalmente, como restricción extra hay que añadir que el tipo de la
        expresión a la izquierda sea un tipo función que tomo el tipo de la
        expresión a la derecha y regresa el tipo de la aplicación.

        \begin{minted}{haskell}
            infer' (nv, (Sintax.App e1 e2)) = 
            let (nv1, g1, t1, r1) = infer' (nv, e1);
                (nv2, g2, t2, r2) = infer' (nv1, e2);
                s = [(s1, s2) | (x, s1) <- g1, (y, s2) <- g2, x == y];
                z = fresh nv2; --item para operadores
                nv' = nv2 `List.union` [z]
                in (
                    nv', 
                    g1 `List.union` g2, 
                    z, 
                    r1 `List.union` r2 `List.union` s `List.union` [(t1, t2 Type.:-> z)
                    )
        \end{minted}

        \item Ligado de variables

        Es similar al procesamiento de funciones.

        Sólo que hay que tomar en cuenta dos subexpresiones en lugar de una.

        \begin{minted}{haskell}
            infer' (nv, (Sintax.Let x e1 e2)) = 
            let (nv1, g1, t1, r1) = infer' (nv, e1);
                (nv2, g2, t2, r2) = infer' (nv1, e2);
                (nv3, g3, t3) =
                case find x g2 of
                    Just t' -> (nv2, filter (\(i, t) -> i /= x) g2, t')
                    Nothing -> (nv2 `List.union` [t'], g2, t') where t' = fresh nv2;
                s = [(s1, s3) | (x, s1) <- g1, (y, s3) <- g3, x == y];
                in (
                    nv3, 
                    g1 `List.union` g3, 
                    t2, 
                    r1 `List.union` r2 `List.union` s `List.union` [(t1, t3)]
                    )
        \end{minted}
    \end{itemize}

    \subsubsection{\texttt{infer}}

    Usando \texttt{infer'}, se obtiene el tipo sin restricciones, las
    restricciones y el contexto.

    Usando la función \texttt{Unifier.µ}, se intenta unificar las restricciones.

    Si esto se logra, se aplica el unificador tanto al contexto como al tipo de
    la expresión, que es lo que regresa la función.

    \begin{minted}{haskell}
        infer :: (Sintax.Expr) -> (Ctxt, Type.Type)
        infer e = 
            let (_, g, t, r) = infer' ([], e); umg = Unifier.µ r
            in (subst g (head umg), Type.subst t (head umg))
    \end{minted}


    \subsection{Integración}

    \subsubsection{Modificaciones a \texttt{Semantic}}

    Parte de la funcionanlidad que se añadió antes pertenecía al este módulo.
    Así que fue necesario partirlo en tres partes. Los módulos \texttt{Static},
    \texttt{Dynamic} y \texttt{Type}.

    Y se convirtió al módulo \texttt{Semantic} en una especie de proxy cuya
    única función es envolver estos tres módulos para no tener que realizar
    modificaciones ni a \texttt{Parser} ni a \texttt{Main}.

    Tanto \texttt{Static} como \texttt{Type} son el tema de la sección anterior.

    Así que sólo falta discutir \texttt{Dynamic}.

    Ahí se guardaron todas las funciones para evaluar expresiones. En estas, se
    tuvo que añadir el caso para el operador \texttt{Fix}, que anteriormente no
    era parte del lenguaje.

    Incluyendo aquellas que hacían uso del análisis estático para revisar los
    tipos antes de realizar la evaluación. Por lo que \texttt{Dynamic} hace uso
    de \texttt{Static}.

    \subsubsection{Modificaciones a \texttt{Sintax}}

    Únicamente se agregaron los casos necesarios para \texttt{Fix}.

    \subsubsection{Modificaciones a \texttt{Main}}

    Al tener \texttt{Type} y \texttt{Static} funciones con el mismo nombre, al
    ser envueltas en \texttt{Semantic} tuvieron que ser importados de manera
    calificada.

    Así que al usar recursos de este módulo en \texttt{Main}, hubo que envolver
    los nombres.

    \section{Dificultades}

    La única gran dificultad fue la parte del algoritmo de inferencia de tipos
    que teníamos que implementar.

    \subsection{\texttt{infer'}}

    Hubo que tener mucho cuidado con el manejo de los contextos en las funciones
    y con el operador \texttt{let}.

    Además, encontrar al manera adecuada de procesar el operador \texttt{if} fue
    tardado, al no haber un ejemplo sobre operadores ternarios en la descripción
    de la práctica.

    Además, surgió el inconveniente que es posible tener varios tipos para una
    misma variable en un mismo contexto, así que para borrarla no siempre bastaba
    con eliminar la primera aparición.


\end{document}